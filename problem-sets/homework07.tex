\documentclass[12pt]{pset}

\course{Math 5801}
\author{Jim Fowler}
\date{Autumn 2015}
\usepackage{amsmath}
\usepackage{tikz}
\usetikzlibrary{arrows,chains,matrix,positioning,scopes}

\makeatletter
\tikzset{join/.code=\tikzset{after node path={%
\ifx\tikzchainprevious\pgfutil@empty\else(\tikzchainprevious)%
edge[every join]#1(\tikzchaincurrent)\fi}}}
\makeatother

\tikzset{>=stealth',every on chain/.append style={join},
         every join/.style={->}}

\newcommand{\NN}{\mathbb{N}}
\newcommand{\Z}{\mathbb{Z}}
\newcommand{\Q}{\mathbb{Q}}
\newcommand{\RR}{\mathbb{R}}
\newcommand{\CC}{\mathbb{C}}
\newcommand{\CP}{\mathbb{C}P}
\newcommand{\RP}{\mathbb{R}P}

\usepackage{enumerate}
\usepackage{nopageno}
\usepackage{hyperref}
\usepackage{mathrsfs}
\newcommand{\TT}{\mathscr{T}}

\DeclareMathOperator{\Ext}{Ext}
\DeclareMathOperator{\Hom}{Hom}

\begin{document}
\maketitle

\noindent The exercises below should be handed in on \textit{Monday,
  October 19, 2015}.  By providing well-written solutions to the
problems, you may earn up \textbf{six points.}  I encourage you to
\textbf{work together} when solving the problems, but please write
your answers independently and include the names of your
co-conspirators.

%%%%%%%%%%%%%%%%%%%%%%%%%%%%%%%%%%%%%%%%%%%%%%%%%%%%%%%%%%%%%%%%
\begin{problem}[Compact and closed]

  Suppose $X$ is a compact topological space, and $K \subset X$ is a
  compact subset.  Is $K$ necessarily closed?  Is there a natural
  condition on $X$ which guarantees that $K$ is closed?

\end{problem}

%%%%%%%%%%%%%%%%%%%%%%%%%%%%%%%%%%%%%%%%%%%%%%%%%%%%%%%%%%%%%%%%
\begin{problem}[Compacts and unions]

  Suppose $A$ and $B$ are compact subsets of a topological space $X$.
  Is it necessarily the case that $A \cup B$ is compact?  If so, prove
  it; if not, find a counterexample.

\end{problem}

%%%%%%%%%%%%%%%%%%%%%%%%%%%%%%%%%%%%%%%%%%%%%%%%%%%%%%%%%%%%%%%%
\begin{problem}[Compact metric spaces]

  Suppose $(X,d)$ is a metric space, and suppose for all
  $\epsilon > 0$ that $X$ can be covered by finitely many open balls
  of radius $\epsilon$.   Must $X$ be compact?
  
\end{problem}

\clearpage

%%%%%%%%%%%%%%%%%%%%%%%%%%%%%%%%%%%%%%%%%%%%%%%%%%%%%%%%%%%%%%%%
\begin{problem}[Compact spaces and products]

  Suppose $X$ and $Y$ are compact topological spaces.  Prove that
  $X \times Y$ is compact.

\end{problem}

%%%%%%%%%%%%%%%%%%%%%%%%%%%%%%%%%%%%%%%%%%%%%%%%%%%%%%%%%%%%%%%%
\begin{problem}[Compact versus sequentially compact]

  Let $I = [0,1]$.  Then the space $X = \{0,1\}^I$ is compact being
  the product of compact spaces.  Find a sequence in $X$ which does
  not converge.

  (\textit{Hint:} define $a_n(x)$ to be the $n$th binary bit of $x$.)

\end{problem}

%%%%%%%%%%%%%%%%%%%%%%%%%%%%%%%%%%%%%%%%%%%%%%%%%%%%%%%%%%%%%%%%
\begin{problem}[Locally compact]

  Suppose $X$ and $Y$ are Hausdorff spaces, and suppose $X$ is locally
  compact (which for right now will mean that every point of $x$ is
  contained in some open set with compact closure).  If $f : X \to Y$
  is a continuous, open, and onto function, then is $Y$ locally
  compact?
  
\end{problem}

\end{document}
