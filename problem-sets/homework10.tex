\documentclass[12pt]{pset}

\course{Math 5801}
\author{Jim Fowler}
\date{Autumn 2015}
\usepackage{amsmath}
\usepackage{tikz}
\usetikzlibrary{arrows,chains,matrix,positioning,scopes}

\makeatletter
\tikzset{join/.code=\tikzset{after node path={%
\ifx\tikzchainprevious\pgfutil@empty\else(\tikzchainprevious)%
edge[every join]#1(\tikzchaincurrent)\fi}}}
\makeatother

\tikzset{>=stealth',every on chain/.append style={join},
         every join/.style={->}}

\newcommand{\NN}{\mathbb{N}}
\newcommand{\Z}{\mathbb{Z}}
\newcommand{\ZZ}{\mathbb{Z}}
\newcommand{\Q}{\mathbb{Q}}
\newcommand{\RR}{\mathbb{R}}
\newcommand{\CC}{\mathbb{C}}
\newcommand{\CP}{\mathbb{C}P}
\newcommand{\RP}{\mathbb{R}P}

\usepackage{enumerate}
\usepackage{nopageno}
\usepackage{hyperref}
\usepackage{mathrsfs}
\newcommand{\TT}{\mathscr{T}}

\DeclareMathOperator{\Ext}{Ext}
\DeclareMathOperator{\Hom}{Hom}

\begin{document}
\maketitle

\noindent The exercises below should be handed in on \textit{Monday,
  December 9, 2015}.  By providing well-written solutions to the
problems, you may earn up \textbf{six points.}  I encourage you to
\textbf{work together} when solving the problems, but please write
your answers independently.

%%%%%%%%%%%%%%%%%%%%%%%%%%%%%%%%%%%%%%%%%%%%%%%%%%%%%%%%%%%%%%%%
\begin{problem}

Are $\R^2$ and $\R^3$ homeomorphic?

\end{problem}

%%%%%%%%%%%%%%%%%%%%%%%%%%%%%%%%%%%%%%%%%%%%%%%%%%%%%%%%%%%%%%%% 
\begin{problem}

  Is the trefoil not tricolorable?  Is the figure~8 knot tricolorable?

\end{problem}

%%%%%%%%%%%%%%%%%%%%%%%%%%%%%%%%%%%%%%%%%%%%%%%%%%%%%%%%%%%%%%%% 
\begin{problem}

  Show that the figure~8 knot is distinct from unknot.

\end{problem}

%%%%%%%%%%%%%%%%%%%%%%%%%%%%%%%%%%%%%%%%%%%%%%%%%%%%%%%%%%%%%%%%
\begin{problem}

  Show that the fundamental group of the complement of two unlinked
  unknots in $S^3$ is nonabelian.

\end{problem}

%%%%%%%%%%%%%%%%%%%%%%%%%%%%%%%%%%%%%%%%%%%%%%%%%%%%%%%%%%%%%%%%
\begin{problem}

  Identify the complement of the Hopf Link in $S^3$ as a product of
  three spaces, and identify the fundamental group of the complement.
  
\end{problem}

%%%%%%%%%%%%%%%%%%%%%%%%%%%%%%%%%%%%%%%%%%%%%%%%%%%%%%%%%%%%%%%%
\begin{problem}

  Draw a picture of a M\"obius band with boundary a trefoil knot.  Can
  you draw a picture of a torus minus a disk with boundary a trefoil
  knot?


\end{problem}






\end{document}
