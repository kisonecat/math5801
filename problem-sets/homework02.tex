\documentclass[12pt]{pset}

\course{Math 5801}
\author{Jim Fowler}
\date{Autumn 2015}
\usepackage{amsmath}
\usepackage{tikz}
\usetikzlibrary{arrows,chains,matrix,positioning,scopes}

\makeatletter
\tikzset{join/.code=\tikzset{after node path={%
\ifx\tikzchainprevious\pgfutil@empty\else(\tikzchainprevious)%
edge[every join]#1(\tikzchaincurrent)\fi}}}
\makeatother

\tikzset{>=stealth',every on chain/.append style={join},
         every join/.style={->}}

\newcommand{\NN}{\mathbb{N}}
\newcommand{\Z}{\mathbb{Z}}
\newcommand{\Q}{\mathbb{Q}}
\newcommand{\RR}{\mathbb{R}}
\newcommand{\CC}{\mathbb{C}}
\newcommand{\CP}{\mathbb{C}P}
\newcommand{\RP}{\mathbb{R}P}

\usepackage{enumerate}
\usepackage{nopageno}
\usepackage{hyperref}
\usepackage{mathrsfs}
\newcommand{\TT}{\mathscr{T}}

\DeclareMathOperator{\Ext}{Ext}
\DeclareMathOperator{\Hom}{Hom}

\begin{document}
\maketitle

\noindent The exercises below should be handed in on \textit{Monday, September
  14, 2015}.  By providing well-written solutions to the problems, you
may earn up \textbf{six points.}  I encourage you to \textbf{work
  together} when solving the problems, but please write your answers
independently and include the names of your co-conspirators.

%%%%%%%%%%%%%%%%%%%%%%%%%%%%%%%%%%%%%%%%%%%%%%%%%%%%%%%%%%%%%%%%
\begin{problem}[Exchanging open and closed sets]

  Let $(X,\TT)$ be a topological space.  Define
  \[
  \TT' = \{ X - U \mid U \in \TT \}.
  \]
  Is it always (never? sometimes?) the case that $(X,\TT')$ is a
  topological space?  \\
  Justify your answer.

\end{problem}

%%%%%%%%%%%%%%%%%%%%%%%%%%%%%%%%%%%%%%%%%%%%%%%%%%%%%%%%%%%%%%%%
\begin{problem}[Combining topological spaces]

  Suppose that for each $i \in I$, we have a topology $\TT_i$ on a set $X$.
  \begin{itemize}
  \item Is $\left( X, \displaystyle\bigcap_{i \in I} \TT_i \right)$ a topological space?
  \item Is $\left( X, \displaystyle\bigcup_{i \in I} \TT_i \right)$ a topological space?
  \end{itemize}
  If so, prove it.  If not, find a counterexample.

\end{problem}

%%%%%%%%%%%%%%%%%%%%%%%%%%%%%%%%%%%%%%%%%%%%%%%%%%%%%%%%%%%%%%%%
\begin{problem}[Coarser or finer?]

  Consider a few topologies on $\RR$.
  \begin{itemize}
  \item Let $\TT_1$ be the standard topology on $\RR$.
  \item Let $\TT_2$ be the topology given by
    \[
    \TT_2 = \{ U \in \mathcal{P}(\RR) \mid U = \varnothing \mbox{ or } \RR - U \mbox{ is finite } \}.
    \]
  \item Let $\TT_3$ be the topology given by
    \[
    \TT_3 = \{ U \in \mathcal{P}(\RR) \mid U = \varnothing \mbox{ or } \RR - U \mbox{ is countable } \}.
    \]
  \item Let $\TT_4$ be the topology on $\RR$ generated by intervals of the
    form $(a,b]$ where $a < b$.
  \end{itemize}
  Are any of these topologies comparable?  Which is coarser?  Finer?

\end{problem}

\pagebreak

%%%%%%%%%%%%%%%%%%%%%%%%%%%%%%%%%%%%%%%%%%%%%%%%%%%%%%%%%%%%%%%%
\begin{problem}[The K-topology]

  Let $K = \{ 1/n \mid n \in \NN \}$.  Then consider the topology on
  $\RR$ generated by open intervals $(a,b)$ and all sets of the
  form $(a,b) - K$.  We'll denote this topological space by $\RR_K$.
  Is the set $K$ open in $\RR_K$?  Is it closed?

\end{problem}

%%%%%%%%%%%%%%%%%%%%%%%%%%%%%%%%%%%%%%%%%%%%%%%%%%%%%%%%%%%%%%%%
\begin{problem}[The unit interval]

  Consider $\RR$ with its usual topology.  Is it possible to find
  infinitely many open intervals $U_i = (a_i,b_i)$ so that
  $[0,1] \subset \displaystyle\bigcup_{i \in \NN} U_i$ but for finite $N$,
  \[
  [0,1] \not\subset \bigcup_{i = 1}^N U_i?
  \]
  If so, exhibit an example.  If not, prove it is impossible.

  \vspace{1ex}
  \noindent\textit{Hint:}  Consider the set 
  \[
  X = \{ x \in [0,1] \mid \mbox{There exists $N$ so that } [0,x] \subset \bigcup_{i = 1}^N U_i \}.
  \]
  and then consider the supremum of elements of $X$.

\end{problem}

%%%%%%%%%%%%%%%%%%%%%%%%%%%%%%%%%%%%%%%%%%%%%%%%%%%%%%%%%%%%%%%%
\begin{problem}[More K-topology]

  This problem is a mash-up of the previous two problems. Consider
  again the topological space $\RR_K$.  Is it possible to find
  infinitely many open sets $U_i$ so that
  $[0,1] \subset \displaystyle\bigcup_{i \in \NN} U_i$ but for finite
  $N$,
  \[
  [0,1] \not\subset \bigcup_{i = 1}^N U_i?
  \]
  If so, exhibit an example. If not, prove it is impossible.

\end{problem}


\end{document}
