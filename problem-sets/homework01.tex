\documentclass[12pt]{pset}

\course{Math 5801}
\author{Jim Fowler}
\date{Autumn 2015}
\usepackage{amsmath}
\usepackage{tikz}
\usetikzlibrary{arrows,chains,matrix,positioning,scopes}

\makeatletter
\tikzset{join/.code=\tikzset{after node path={%
\ifx\tikzchainprevious\pgfutil@empty\else(\tikzchainprevious)%
edge[every join]#1(\tikzchaincurrent)\fi}}}
\makeatother

\tikzset{>=stealth',every on chain/.append style={join},
         every join/.style={->}}

\newcommand{\Z}{\mathbb{Z}}
\newcommand{\Q}{\mathbb{Q}}
\newcommand{\CC}{\mathbb{C}}
\newcommand{\CP}{\mathbb{C}P}
\newcommand{\RP}{\mathbb{R}P}

\usepackage{enumerate}
\usepackage{nopageno}
\usepackage{hyperref}

\DeclareMathOperator{\Ext}{Ext}
\DeclareMathOperator{\Hom}{Hom}

\begin{document}
\maketitle

\noindent\textbf{\href{http://en.wikipedia.org/wiki/List_of_games_with_concealed_rules}{The only way to learn the game is to play the game.}}
The following represents a \textit{lower bound} on the number of
exercises you should be doing; the textbook is packed full of great
exercises, so I encourage you to do as many as possible.

The exercises below should be handed in on \textit{Monday, September
  7, 2015}---which is a holiday, so feel free to drop them off a day
later.  By providing well-written solutions to the problems, you may
earn up \textbf{six points.}  I encourage you to \textbf{work
  together} when solving the problems, but please write your answers
independently and include the names of your collaborators.

%%%%%%%%%%%%%%%%%%%%%%%%%%%%%%%%%%%%%%%%%%%%%%%%%%%%%%%%%%%%%%%%
\begin{problem}[Eliminating axioms]

  Suppose $\sim $ is a relation so that for all $x$ and $y$ and $z$, we have
  \begin{align*}
    x \sim  y &\Leftrightarrow y \sim  x  \mbox{ and} \\
    \left(x \sim  y\right) \wedge \left( y \sim  z \right) &\Rightarrow x \sim  z.
  \end{align*}
  We then prove that $x \sim x$ by noting that $x \sim y$ implies
  $y \sim x$, and so $x \sim y \sim x$, and therefore by transitivity,
  $x \sim x$.  So ``reflexivity'' is an unneeded axiom for equivalence
  relations---or is it?  Explain why this is or is not a valid proof
  of reflexivity from transitivity and symmetry.

\end{problem}

%%%%%%%%%%%%%%%%%%%%%%%%%%%%%%%%%%%%%%%%%%%%%%%%%%%%%%%%%%%%%%%%
\begin{problem}[Families of subsets]

  Suppose $A$ consists of all arithmetic progressions in $\mathbb{Z}$,
  i.e., an element of $A$ is a set 
  \[
  S_{m,b} = \{ mx + b \mid x \in \Z \},
  \]
  so $A = \{ S_{m,b} \mid m, b \in \Z \}$.
  Is $A$ closed under unions?  Under intersections?  More precisely,
  if $X, Y \in A$, then is $X \cup Y \in A$?  Is $X \cap Y \in A$?
  Prove your conjectures or provide counterexamples.

\end{problem}

%%%%%%%%%%%%%%%%%%%%%%%%%%%%%%%%%%%%%%%%%%%%%%%%%%%%%%%%%%%%%%%%
\begin{problem}[Countability]

  Let $X = \{0,1\}^{\omega}$, and define
  \[
  Y = \{ (x_1,x_2,\ldots) \in X \mid \mbox{all but finitely many of the $x_i$'s are 0} \}.
  \]
  Prove that $X$ and $Y$ are infinite by exhibiting a bijection
  between $X$ and a proper subset of $X$ and between $Y$ and a proper
  subset of $Y$.  Is there a bijection between $X$ and $Y$?

\end{problem}

%%%%%%%%%%%%%%%%%%%%%%%%%%%%%%%%%%%%%%%%%%%%%%%%%%%%%%%%%%%%%%%%
\begin{problem}[Rigs versus rings]

  Suppose $X$ is a set and suppose $f$ is a bijection
  \[
  f : X \to \{ \varnothing \} \sqcup X \sqcup X^2.
  \]
  By $\sqcup$, I mean ``disjoint union''---emphasizing that $X$
  and $X^2$ and $\{ \varnothing \}$ are meant to be disjoint.  Is there a bijection
  \[
  g : X \sqcup X^3 \to X^2 \sqcup X^4 \,?
  \]
  Can you describe $g$ using $f$?

  %http://math.ucr.edu/home/baez/week202.html
\end{problem}

%%%%%%%%%%%%%%%%%%%%%%%%%%%%%%%%%%%%%%%%%%%%%%%%%%%%%%%%%%%%%%%%
\begin{problem}[Order]

  Regard $\{1,2,3\}$ and $\N$ as ordered sets using the usual
  ordering, and consider
  \begin{align*}
    A &= \{1,2,3\} \times \{1,2,3\} \\
    B &= \N \times \{1,2,3\} \\
    C &= \{1,2,3\} \times \N \\
    D &= \N \times \N
  \end{align*}
  with the lexicographic ordering.  Are there bijections between any
  of the four sets $A$, $B$, $C$, and $D$?  Are there
  \textit{order-preserving} bijections between any of these sets?

\end{problem}


%%%%%%%%%%%%%%%%%%%%%%%%%%%%%%%%%%%%%%%%%%%%%%%%%%%%%%%%%%%%%%%%
\begin{problem}[Subsets]

  A set $A$ of subsets of $\Z$ is \textit{good} if whenever
  $S \subset \Z$ then either $S \in A$ or $\Z - S \in A$ but not both.  For example,
  \[
  A = \{ S \in \mathcal{P}(\Z) \mid 1 \in S \}
  \]
  is good.  A good set of subsets contains, for instance, either the
  set of evens or the set of odds, but not both.

  A set $A$ of subsets of $\Z$ is said to \textit{avoid finite sets} if
  whenever $S$ is a finite set, then $S \not\in A$.  For example, 
  \[
  A = \{ \Z - \{ x \} \mid x \in \Z \}
  \]
  avoids finite sets---but it's not good, because neither the even
  numbers nor the odd numbers are contained in $A$.

  Is there a set of subsets of $\Z$ with both the above properties, i.e.,
  a good set of subsets of $\Z$ which avoids finite sets?
 

\end{problem}



\end{document}
